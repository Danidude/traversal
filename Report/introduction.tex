\chapter{Introduction}
\label{ch:introduction}

\section{Background}

Pathfinding algorithms are used to find the quickest path between two points. 
These algorithms are for instance used to solve mazes, scheduling a long trip in
metro systems or wire routing. In our project we have looked at using pathfinding
algorithms to find the safest route off a ship that is on fire.

Every year people die in boating accidents and we hope that our work can be
used to save lives. Jim Hall, the Chairman of the National Transportation 
Safety Board testified in front of the Subcommittee on Coast Guard and Maritime 
Transportation, making several safety recomendations after the Scandinavian Star
accident \cite{ntsb}. One of them were to "Improve crew language/communication 
ability to assist passengers during emergencies." Our work, which will be part of a system
that provides passengers with the quickest route to the exit, could output 
any directions in the language of the phone owner. This would not only improve 
communication if the employees and passengers do not speak the same language, 
it would also ensure that the directives reach all passengers quicker than if the employees 
were to guide all passengers. Additionally it increases the probability that all passengers 
receives proper guidance. For instance any message played over a sound system could be
difficult to hear during a panic and passengers could be in rooms or corridors
where they would be unrachable for the employees.

\section{Problem Statement}

In this project we set out to find a safe route
to evacuate passengers in crisis situations. As there are a multitude of pathfinding algorithms
we chose to focus on exploring the validity of a single algorithm called Ant Colony Optimization 
(ACO). We chose this algorithm since we believed the nature of the algorithm made it a perfect
candidate in a dynamic environment. Thus, our goal was to use ACO to find safe passage for each passenger
in a crisis situation. To achive this we needed to create a testing environment in which 
we would run the ACO algorithm and then evaluate its performance. 

\subsection{Limitations}
To reach the project goal within the deadline some limitations were set. Firstly, the simulations did not use 
any real data. Secondly, the graphs were completely randomly generated and the graphs did not resemble
any ships or their actual structure. Thirdly, all rooms were equally easy to traverse and had unlimited capasity. 
Finally, there were no efforts to simulate actual human behavior during a crisis and all passengers followed
any directions perfectly.

\subsection{Assumptions}
Additionally we made several assumptions. Firstly, we asumed that a ship could be sufficently modeled
as a graph. Secondly, everyone on the ship had access to a smart phone with an application that ran our
algorithms and they followed the directions perfectly. Finally, all passengers acted the same and moved at
the same speed.

\section{Solution Approach}

Normally, pathfinding algorithms finds the quickest path from a location to an 
exit. However in our case certain sections of the ship could be dangerous and the algorithms 
had to find alternate routes, valuing safety over speed. The ships were modeled as undirected
graphs with nodes and vertices. The nodes represented any type of room, including
hallways, stairs etc., and a vertex between two nodes indicated that it was possible to
cross over from one to the other.

ACO attempts to find the optimal route through the ship by mimicking the behavior of ants and ant colonies. 
The performance of the algorithm was tested by creating random graphs with a specific amount of
passengers in them. We would then run ACO multiple times and compare its performance against
random behavior and perfect behavior. In the case of random behavior the passengers would randomly
walk until they either died or found the exit. And to simulate perfect behavior we found every possible
path for every passenger and chose the best one.

\section{Report outline}

Chapter one gives an overview of how pathfinding algorithms work, what problem we are ting to solve with our
project and our approach to doing so. Additionally it explains our motivation for doing it as well as sets some
limitations and assumptions to the process. The second chapter contains the state of the art which we drew
knowledge from to complete our project. ACO is described in detail, both by showing the formulas and describing
the ideas that lead to the creation of the algorithm.

The third chapter outlines how our model and algorithms were implemented. Furthermore it explains why we chose
to implement certain aspects of the model over other alternative solutions. The fourth chapter contains the results
and how we reach those results. It represents the empirical data gathered from the simulations and attempts to
explain the finding.

The fifth and final chapter contains the discussion. It attempts to determine whether ACO is the best algorithm given
our scenario and what steps can be taken in the future to better develop the algorithm and model.


