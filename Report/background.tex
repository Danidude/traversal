\chapter{State of the art}
\label{ch:background}

\section{Ant Colony Optimization}

The Ant Colony Optimization(ACO) algorithm is a probabilistic algorithm that have been used in routing, scheduling, 
subset and machine learning. It was created by Marco Dorigo in 1992 to find an optimal path in a graph, mimicking the
behavior of ant which are searching for food \cite{aco}. In nature, when searching for food, ants walk randomly until
they locate a food source. When an ant stumble upon food it returns to the ant colony, taking the shortest path possible.
While returning to its colony it leaves a trail of pheromones which will attract other ants. These other ants will, by walking
the same path as the initial ant, strengthen the pheromone trail and further increase the possibility that more ants will
take this path to the food source.

While the pheromone trail increases the probability that an ant choses to follow the trail, it does not ensure it.
If several ants reaches the same food source by walking different paths the shortest path will be made the more attractive
one over time. This is a result of pheromone deposits. The shortest path will have the highest pheromone density as more
pheromones can be deposited on the shorter route, the resulting higher pheromone density will increase the possibility that 
other ants will follow the shorter path. 

Over time pheromones evaporates, this prevents the system from ending up with local optimal solutions. 
If the pheromones did not evaporate the path followed by a certain amount of initial ants
would quickly become the only path any ant followed as the pheromone density would increase to the point that no other paths
could ever become more attractive, even if they were the more optimal paths. 

Additionally, pheromone evaporation makes ACO able to adapt to changes. A path can be the optimal path for a while, untill a change
occurs that makes the path unusable. If the pheromones did not evaporate the ants would continue to follow this path indefinately,
never reaching the food source again. However, when the ants are not able to reach the food source they will not deposit any
pheromones and as the pheromones deposited by previous ants evaporates the ants will will be less attracted to the old path
and start exploring new ones.

For the algorithm itself there are two parts, edge selection and pheromone update. Edge selection is the process of choosing
which way to go. Every ant $s$ has a probability $P^k_{ab}$ of moving from location $a$ to location $b$. The probability
is calculated based on the attractiveness $\Lambda_{ab}$  of moving from $a$ to $b$ and the pheromone trail level $T_{ab}$.

\begin{math}
P^k_{ab} = {(T^\alpha_{ab})(\Lambda^\beta_{ab}) \over \sum (T^\alpha_{ab})(\Lambda^\beta_{ab})}
\end{math}



\begin{math}
T_{ab} \leftarrow (1 - \rho)T_{ab} + \sum_k \Delta T^k_{ab}
\end{math}










%In the original Ant system, each iteration runs m ants through the different paths/graph, and creates a solution. When all the m ants have finished, the update to pheromones happens to the %solution the ants have found. Then the algorithm starts the next iteration with the next m ants.

%MAX-MIN Ant system is an improvement to the original in the way that only the best solution is updated or best so far.

%Ant Colony system uses “local pheromones” and “offline pheromones” in finding the best solution to the problem. The local pheromones are used to update each location so that the other ants %that follows are less likely to take the same path as him, and thus creating several different solutions in the iteration. When the iteration is done, the offline pheromones are updated to the %best solution making it more likely to select this one in the future.



